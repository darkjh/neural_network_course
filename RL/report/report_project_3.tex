\documentclass[a4paper, 12pt]{article}
% Options possibles : 10pt, 11pt, 12pt (taille de la fonte)
% oneside, twoside (recto simple, recto-verso)
% draft, final (stade de développement)
\usepackage[utf8]{inputenc} % LaTeX, comprends les accents !
\usepackage[T1]{fontenc} % Police contenant les caractères français
\usepackage{geometry} % Définir les marges
\usepackage[francais]{babel} % Placez ici une liste de langues, la
% dernière étant la langue principale
\usepackage{graphicx}
\usepackage{verbatim}
\usepackage{float}
\usepackage{booktabs}
\usepackage{amsmath}
\usepackage{amsfonts}
\usepackage{multirow}
\geometry{left=2.2cm,right=2.2cm,top=2.5cm,bottom=2.5cm}
% \pagestyle{headings} % Pour mettre des entêtes avec les titres
% des sections en haut de page
%\include[dvips]{graphics}
\title{Reinforcement Learning Project Report \\ \vspace{0.5cm} \large Unsupervised and Reinforcement Learning in Neural Networks}
\author{Han JU, Hao REN}
\date{\today}
\begin{document}
\maketitle
\section{Learning curve}

\section{Exploration \& Exploitation}
The parameter $\epsilon$ controls the balance between exploration and exploitation. When $\epsilon$ is big, the agent has more possibilities to take a random step in order to explore more place in the environement, while small $\epsilon$ makes the agent more likely to take a step according to the $\epsilon$-greedy policy. The implementation of the algorithm SARAR depends on the choice of the $\epsilon$, we have to make a tradeoff. If the $\epsilon$ is too big, the agent will seldom take a step with the greedy policy, the whole learning process is more stochastic, so the learning result is not significant. However, if $\epsilon$ is too small, the agent will be more likely to exploit the state around where it is, because the weights are not largely updated, the agent sometimes will travel around several same states, sometimes even not get to the goal within a given time step limit.

The best solution is to start with a big $\epsilon$ so as to explore the space as widely as possible. The $\epsilon$ decreases after each trails Thus, the first several trials with big $\epsilon$ will collect enough information about the weights (Q values), and the later trials with small $\epsilon$ can take a good use of these information via the greedy policy which makes the agent more likely to get into the goal area.

The learning curves and the average latency of the last 10 trails with $\epsilon$=0.8, 0.6, 0.4, 0.2 are shown as below:

\\figure

$\epsilon$ = 0.8 $\Rightarrow$ 219 steps

$\epsilon$ = 0.6 $\Rightarrow$ 113 steps

$\epsilon$ = 0.4 $\Rightarrow$ 77  steps

$\epsilon$ = 0.2 $\Rightarrow$ 79 steps



\section{Navigation Map}


The naviagation map is shown as below: 

//figure


\end{document}