\documentclass[a4paper, 12pt]{article}

% Options possibles : 10pt, 11pt, 12pt (taille de la fonte)
% oneside, twoside (recto simple, recto-verso)
% draft, final (stade de développement)

\usepackage[utf8]{inputenc} % LaTeX, comprends les accents !
\usepackage[T1]{fontenc} % Police contenant les caractères français
\usepackage{geometry} % Définir les marges
\usepackage[francais]{babel} % Placez ici une liste de langues, la
% dernière étant la langue principale
\usepackage{graphicx}
\usepackage{verbatim}
\usepackage{float}
\usepackage{booktabs}
\usepackage{amsmath}
\usepackage{amsfonts}
\usepackage{multirow}

\geometry{left=2.2cm,right=2.2cm,top=2.5cm,bottom=2.5cm}
% \pagestyle{headings} % Pour mettre des entêtes avec les titres
% des sections en haut de page
%\include[dvips]{graphics}

\title{Kohonen Map Project Report \\ \vspace{0.5cm} \large Unsupervised and Reinforcement Learning in Neural Networks}

\author{Han JU, Hao Ren}
\date{\today}

\begin{document}
\maketitle

\section{Learning rate and convergence}

\section{Assignment method}
In this step, we need a method to automatically assign the propre
digit to each prototype that we found by executing the Kohonen map
algorithm. An obvious idea is to look at the labels of the trainning
data points that are close to the prototype, then decide the its label
accroding to the majority in this surrounding. In fact, this task has
a similarity with the $k$ nearest neighbor classification algorithm,
which assigns a label to a data sample based on the majority of its $k$
neighbors. So we decided to use the kNN method.

Before we apply this method directly to the problem, however, we need
to determine the parameter, $k$, of this algorithm. Intuitively, this
parameter depends on our trainning data, which covers the hand
writing samples of the four chosen digits. So we do firstly a cross
validation on kNN models with distinct $k$ values (ranging from 3 to 300),
which results in $k$ equals 5.

Once the paramter is fixed, the application of kNN to the problem is fairly straightforward.

\section{Comparing the results}

\section{Conclusion}

\end{document}